\chapter{Signals and Systems}\label{ch:1}

1.2 In signal processing, sampling is the reduction of a continuous-time signal to a discrete-time signal. A common example is the conversion of a sound wave (a continuous signal) to a sequence of samples (a discrete-time signal).\cite{wiki:Sampler}
so, basically a sampler records the signal value at particular time intervals. Here the sampler records the signal $\sin(t)$ at intervals of $k\pi/4$ where k is an integer.

\begin{figure}[h]
	\centering
	\includegraphics[width=0.7\linewidth]{graphics/sampler}
	\caption[Sampler response]{The Response of the sampler is plotted}
	\label{fig:sampler}
\end{figure}

1.5 (c)
 
\[z(s) = 3 + \dfrac{s}{s^{2}+2}\]
\[= 3 + \frac{1}{s + \frac{2}{s}}\]
The 3 represents a resistor of value $3\Omega$ in series with another impedance $z_{eq}$
\[z_{eq} = \frac{1}{\frac{1}{z_{1}}+ \frac{1}{z_{2}}} \] parallel connection of $z_{1}$ and $z_{2}$
\[\frac{1}{z_{eq}(s)} = \frac{1}{z_{1}} + \frac{1}{z_{2}}\]
$\therefore$ 
\[z_{1} = \frac{1}{s} ,\;\;\;\; z_{2} = \frac{s}{2}\]

\begin{figure}[h]
	\centering
	\includegraphics[width=0.7\linewidth]{"/circuit 1_5(c)"}
	\caption[Network Realization for z(s)]{Network Realization for z(s)}
	\label{fig:circuit-1_5(c)}
\end{figure}

1.5 (d)
\[y(s) = \frac{1}{3s+2} + \frac{2s}{s^{2} + 4}\]
\[\frac{1}{z(s)}= \frac{1}{3s + 2} + \frac{1}{\frac{s^{2}+4}{2s}} \]
which is similar to \[\frac{1}{z_{eq}(s)} = \frac{1}{z_{1}} + \frac{1}{z_{2}}\]
\[z_{1} = 3s+2 \;\;\;\; z_{2} = \frac{s}{2}  + \frac{2}{s} \]

\begin{figure}[h]
	\centering
	\includegraphics[width=0.7\linewidth,height=0.43\linewidth]{"/circuit 1_5(d)"}
	\caption{Network realization for y(s)}
	\label{fig:circuit-15d}
\end{figure}



\bibliography{backmatter/mybib}